%%%%%%%%%%%%%%%%%%%%%%%%%%%%%%%%%%%%%%%%%%%%%%%%%%%%%%%%%%%%%%%%%%%%%%%%%%% 
% 
% Generic template for the anteproyectos of TFC/TFM/TFGs
% 
% $Id: anteproyecto.tex,v 1.6 2018/09/11 12:23:48 macias Exp $
% 
% By:
%  + Javier Macías-Guarasa. 
%    Departamento de Electrónica
%    Universidad de Alcalá
%  + Roberto Barra-Chicote. 
%    Departamento de Ingeniería Electrónica
%    Universidad Politécnica de Madrid   
% 
% Based on original sources by Roberto Barra, Manuel Ocaña, Jesús Nuevo,
% Pedro Revenga, Fernando Herránz and Noelia Hernández. Thanks a lot to
% all of them, and to the many anonymous contributors found (thanks to
% google) that provided help in setting all this up.
% 
% See also the additionalContributors.txt file to check the name of
% additional contributors to this work.
% 
% If you think you can add pieces of relevant/useful examples,
% improvements, please contact us at (macias@depeca.uah.es)
% 
% You can freely use this template and please contribute with
% comments or suggestions!!!
% 
%%%%%%%%%%%%%%%%%%%%%%%%%%%%%%%%%%%%%%%%%%%%%%%%%%%%%%%%%%%%%%%%%%%%%%%%%%% 

% This is for rubber to clean additional files
% rubber: clean anteproyecto.acn anteproyecto.acr anteproyecto.alg anteproyecto.cod anteproyecto.ist anteproyecto.out anteproyecto.sbl anteproyecto.slg anteproyecto.sym anteproyecto.lor

%%%%%%%%%%%%%%%%%%%%%%%%%%%%%%%%%%%%%%%%%%%%%%%%%%%%%%%%%%%%%%%%%%%%%%%%%%% 
% BEGIN Preamble and configuration section
% 
\input{../Config/preamble-anteproyecto.tex}    % DO NOT TOUCH THIS LINE. You can edit
% the file to modify some default settings

\input{../Config/myconfig.tex}    % DO NOT TOUCH THIS LINE, but EDIT THIS FILE 
                                  % to set your specific settings (related
                                  % to the document language, your degree,
                                  % document details (such as title, author
                                  % (you), your email, name of the tribunal
                                  % members, document year, keyword and
                                  % palabras clave) and link colors), and
                                  % define your commonly used commands
                                  % (some examples are provided).

\input{../Config/postamble-anteproyecto.tex}   % DO NOT TOUCH THIS LINE. Yes, I know,
                                  % "postamble" is not a valid word... :-)

% path to directories containing images
\graphicspath{{../Book/logos/}{../Book/figures/}{../Book/diagrams/}} % Edit this to your
                                  % needs. Only logos is really required
                                  % when you generate your own content.
% 
% END Preamble and configuration section
%%%%%%%%%%%%%%%%%%%%%%%%%%%%%%%%%%%%%%%%%%%%%%%%%%%%%%%%%%%%%%%%%%%%%%%%%%% 

\title{Anteproyecto de \myWorkTypeFull}        % DO NOT TOUCH THIS LINE
\date{\myThesisProposalDate}                         % DO NOT TOUCH THIS LINE
\author{\myAuthorFullName}

%%%%%%%%%%%%%%%%%%%%%%%%%%%%%%%%%%%%%%%%%%%%%%%%%%%%%%%%%%%%%%%%%%%%%%%%%%% 
% Let's start with the real stuff
%%%%%%%%%%%%%%%%%%%%%%%%%%%%%%%%%%%%%%%%%%%%%%%%%%%%%%%%%%%%%%%%%%%%%%%%%%% 
\begin{document}                                   % DO NOT TOUCH THIS LINE

%%%%%%%%%%%%%%%%%%%%%%%%%%%%%%%%%%%%%%%%%%%%%%%%%%%%%%%%%%%%%%%%%%%%%%%%%%% 
% BEGIN within-document configuration, frontpage and cover pages generation
% 

% Set Language dependent issues that must be set after \begin{document}
\input{../Config/setlanguagedependentissues.tex} % DO NOT TOUCH THIS LINE
                                                 % NOR THE FILE

% 
% END within-document configuration, frontpage and cover pages generation
%%%%%%%%%%%%%%%%%%%%%%%%%%%%%%%%%%%%%%%%%%%%%%%%%%%%%%%%%%%%%%%%%%%%%%%%%%% 

\maketitle

\begin{description}                               % DO NOT TOUCH THIS LINE
\item[\expandafter\makefirstuc\expandafter{\wordAutorOrAutora}:] \myAuthorFullName                       % DO NOT TOUCH THIS LINE
\item[\expandafter\makefirstuc\expandafter{\wordTutorOrTutores}:] \myAdvisors                   % DO NOT TOUCH THIS LINE
  \item[Titulación:] \myDegreefull                % DO NOT TOUCH THIS LINE
  \item[Título:] \myBookTitleSpanish              % DO NOT TOUCH THIS LINE
  \ifthenelse{\equal{\myLanguage}{english}}       % DO NOT TOUCH THIS LINE
  {                                               % DO NOT TOUCH THIS LINE
  \item[Título en inglés:] \myBookTitleEnglish    % DO NOT TOUCH THIS LINE
  }                                               % DO NOT TOUCH THIS LINE
  {                                               % DO NOT TOUCH THIS LINE
  }                                               % DO NOT TOUCH THIS LINE
\item[Departamento:] \myDepartment                % DO NOT TOUCH THIS LINE
\end{description}                                 % DO NOT TOUCH THIS LINE

%\tableofcontents

%%%%%%%%%%%%%%%%%%%%%%%%%%%%%%%%%%%%%%%%%%%%%%%%%%%%%%%%%%%%%%%%%%%%%%%%%%% 
%%%%%%%%%%%%%%%%%%%%%%%%%%%%%%%%%%%%%%%%%%%%%%%%%%%%%%%%%%%%%%%%%%%%%%%%%%% 
%%%%%%%%%%%%%%%%%%%%%%%%%%%%%%%%%%%%%%%%%%%%%%%%%%%%%%%%%%%%%%%%%%%%%%%%%%% 
%%%%%%%%%%%%%%%%%%%%%%%%%%%%%%%%%%%%%%%%%%%%%%%%%%%%%%%%%%%%%%%%%%%%%%%%%%% 
%%%%%%%%%%%%%%%%%%%%%%%%%%%%%%%%%%%%%%%%%%%%%%%%%%%%%%%%%%%%%%%%%%%%%%%%%%% 
%%%%%%%%%%%%%%%%%%%%%%%%%%%%%%%%%%%%%%%%%%%%%%%%%%%%%%%%%%%%%%%%%%%%%%%%%%% 
%%%%%%%%%%%%%%%%%%%%%%%%%%%%%%%%%%%%%%%%%%%%%%%%%%%%%%%%%%%%%%%%%%%%%%%%%%% 
% BEGIN Normal sections. Edit/modify all within this section

\section{Introducción}
\label{sec:introduccion}

En los últimos años, la detección de objetos en tres dimensiones (3D) ha ganado un rol central en el desarrollo de vehículos autónomos, convirtiéndose en un desafío crucial para mejorar la precisión y seguridad en la conducción automatizada. La combinación de sensores LiDAR y cámaras ha permitido capturar tanto información de profundidad como de color y textura, aspectos esenciales para la interpretación completa del entorno en el que se desenvuelven los sistemas autónomos. Este Trabajo de Fin de Grado (TFG) se enfoca en la detección de objetos 3D mediante la integración de métodos de deep learning y fusion temporal aplicados a imágenes obtenidas de cámaras y datos de sensores LiDAR.

El proyecto utiliza el modelo YOLO (You Only Look Once) para la detección de objetos, principalmente vehículos, en un entorno dinámico captado por un coche autónomo. Mediante el uso de PyTorch, se entrena y optimiza la estimación tridimensional de YOLO, permitiendo una comprensión espacial avanzada de los elementos presentes en el entorno de conducción. Además, para alcanzar una representación temporal coherente y precisa, se propone una fusión de datos basada en la fusion temporal de distintas secuencias de imágenes de cámara y datos LiDAR, capturadas en diferentes momentos y posiciones. Esta técnica de fusión temporal tiene como objetivo lograr una detección más robusta y confiable, especialmente en condiciones de tráfico y situaciones complejas.

Para el desarrollo y la evaluación del modelo, se recurrirá a datasets ampliamente utilizados en el campo de la conducción autónoma, como KITTI. La implementación de estas técnicas permitirá obtener una detección más precisa de los objetos 3D alrededor de un vehículo, contribuyendo al desarrollo de sistemas de conducción autónoma más avanzados y seguros.

\section{Objetivos}
\label{sec:objetivos-y-campo}


El objetivo fundamental de este proyecto es el desarrollo e implementación de un sistema de detección y estimación de objetos en 3D utilizando redes neuronales de deep learning, específicamente el modelo YOLO, en combinación con un esquema de fusión temporal de datos provenientes de sensores de cámara y LiDAR, montados en un vehículo autónomo. Este sistema busca mejorar la precisión en la detección y localización tridimensional de objetos en el entorno, en especial vehículos, contribuyendo a la eficiencia y seguridad en la conducción autónoma.


Los objetivos específicos de este proyecto son los siguientes:
\begin{itemize}
	\item Estudiar la arquitectura de YOLO y su adaptación a la detección 3D:
	\begin{enumerate}
		\item Realizar una revisión de la arquitectura YOLO y analizar su capacidad para detectar y estimar objetos en 3D a partir de imágenes 2D.
		\item Implementar la versión de YOLO optimizada en PyTorch para la estimación de profundidad, ajustando sus características para mejorar el reconocimiento tridimensional de objetos.
	\end{enumerate}
	
	\item Diseñar e implementar un esquema de fusión temporal de datos de cámaras y LiDAR:
	\begin{enumerate}
		\item Desarrollar un modelo de fusión de datos que integre las estimaciones en tiempo real de los sensores de cámara y LiDAR.
		\item Explorar técnicas de fusión temporal para mejorar la precisión y consistencia en la detección de objetos en el entorno, utilizando información recopilada en intervalos de tiempo sucesivos.
	\end{enumerate}
	
	\item Entrenar y optimizar el modelo YOLO en un entorno de conducción autónoma:
	\begin{enumerate}
		\item Experimentar con diferentes técnicas de entrenamiento para optimizar el rendimiento del modelo en la detección y clasificación de objetos 3D, tomando en cuenta las limitaciones de la inferencia en tiempo real.
	\end{enumerate}
	
	\item Evaluar el rendimiento del sistema de detección y fusión temporal:
	\begin{enumerate}
		\item Medir la precisión y rapidez de detección del sistema evaluandolo en condiciones de simulación y con datos de entornos reales.
	\end{enumerate}
	
	\item Implementar mejoras de precisión y reducir los tiempos de inferencia del modelo:
	\begin{enumerate}
		\item Proponer ajustes en el modelo de YOLO y en la fusión de datos para optimizar la precisión y minimizar el tiempo de respuesta.
		\item Desarrollar herramientas de análisis de error y realizar pruebas con datasets específicos de conducción autónoma para lograr mejoras consistentes en el rendimiento.
	\end{enumerate}
\end{itemize}

\begin{figure}[hbtp]
	\centering
	\includegraphics[width=0.6\textwidth]{flujo.png}
	\caption{Flujo del sistema de detección y estimación 3D.}
	\label{fig:flujo_sistema}
\end{figure}



\section{Metodología y plan de trabajo}
\label{sec:metodologia-y-plan}

\textit{Aquí se incluirá una descripción (puede ser incluso una enumeración)
  clara de las etapas que se van a seguir, y si es posible se deberá
  incluir un diagrama de Gantt. Por ejemplo algo del estilo a:}

Estas son las fases de desarrollo que se van a seguir para la
consecución de los objetivos del proyecto descritos en la sección~\ref{sec:objetivos-y-campo}:

\begin{enumerate}
  
\item Formación inicial (1 mes)
  
  \begin{itemize}
  \item Formación en \ldots
  \item Consulta de la API \ldots
  \item Consulta bibliográfica \ldots
  \item Profundización en herramientas de soporte \ldots
  \item  \ldots
  \end{itemize}

\item Diseño del entorno \ldots (0,5 meses)

\item Diseño, implementación y evaluación del  \ldots (2 meses)
  \begin{itemize}
  \item Definición del  \ldots
  \item Implementación del  \ldots
  \item Evaluación del  \ldots
  \end{itemize}
  
\item Implementación y evaluación de  \ldots (1 mes)

\item Diseño, implementación y evaluación del módulo  \ldots (1 mes):
  \begin{itemize}
  \item Definición de  \ldots
  \item Definición de  \ldots
  \item Implementación y evaluación  \ldots
  \end{itemize}

\item Integración y  \ldots (1 mes)

\item Documentación  \ldots (0,5 meses)

\end{enumerate}


\section{Medios}
\label{sec:medios}

\textit{Aquí se describen de los medios necesarios para realizar el TFG. Por
  ejemplo:}

Las herramientas que van a ser necesarias para desarrollar este proyecto
son las siguientes:

\begin{itemize}
\item PC compatible
\item Sensor  \ldots
\item Sistema operativo GNU/Linux~\cite{gnulinux}
\item Entorno de desarrollo Emacs/Vim~\cite{emacs}
\item Procesador de textos \LaTeX~\cite{lamport94}
\item Control de versiones CVS~\cite{cvs}
\item Compilador C/C++ gcc~\cite{gcc}
\item Gestor de compilaciones make~\cite{make}
\item Robot con movilidad.
\item  \ldots
\end{itemize}



Otros recursos necesarios para la elaboración del proyecto son:

\begin{itemize}
\item Herramientas  \ldots
\item Sistema de desarrollo  \ldots
\item  \ldots
\end{itemize}




% 
% END Normal sections. Edit/modify all within this section
%%%%%%%%%%%%%%%%%%%%%%%%%%%%%%%%%%%%%%%%%%%%%%%%%%%%%%%%%%%%%%%%%%%%%%%%%%% 
%%%%%%%%%%%%%%%%%%%%%%%%%%%%%%%%%%%%%%%%%%%%%%%%%%%%%%%%%%%%%%%%%%%%%%%%%%% 
%%%%%%%%%%%%%%%%%%%%%%%%%%%%%%%%%%%%%%%%%%%%%%%%%%%%%%%%%%%%%%%%%%%%%%%%%%% 
%%%%%%%%%%%%%%%%%%%%%%%%%%%%%%%%%%%%%%%%%%%%%%%%%%%%%%%%%%%%%%%%%%%%%%%%%%% 
%%%%%%%%%%%%%%%%%%%%%%%%%%%%%%%%%%%%%%%%%%%%%%%%%%%%%%%%%%%%%%%%%%%%%%%%%%% 
%%%%%%%%%%%%%%%%%%%%%%%%%%%%%%%%%%%%%%%%%%%%%%%%%%%%%%%%%%%%%%%%%%%%%%%%%%% 
%%%%%%%%%%%%%%%%%%%%%%%%%%%%%%%%%%%%%%%%%%%%%%%%%%%%%%%%%%%%%%%%%%%%%%%%%%% 


%%%%%%%%%%%%%%%%%%%%%%%%%%%%%%%%%%%%%%%%%%%%%%%%%%%%%%%%%%%%%%%%%%%%%%%%%%% 
% Bibliography
%%%%%%%%%%%%%%%%%%%%%%%%%%%%%%%%%%%%%%%%%%%%%%%%%%%%%%%%%%%%%%%%%%%%%%%%%%% 
\input{../Book/biblio/bibliography.tex}               % EDIT this file if required



\end{document}

